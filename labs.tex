% A basic LaTeX document for an assignment

% If you want the title to appear on a separate
% page, change notitlepage to titlepage
\documentclass[11pt,a4paper,notitlepage]{article}
\usepackage[utf8]{inputenc}
\usepackage[T1]{fontenc}
% If your hand-in is in icelandic change english to icelandic
% Note: This has nothing to do with Icelandic characters, they
% can always be used. This just tells other packages what 
% language you are using and changes the hyphenation used by LaTeX
% If icelandic is selected, a shorthand, "` and "', is also included
% for Icelandic quotation marks. They can also obtained by using 
% ,, and ``
\usepackage[english]{babel}
\usepackage{amsmath, amsthm, amssymb, amsfonts}
\usepackage{graphicx}
\usepackage{enumerate}
% To use the whole A4-page
% See: ftp://ftp.tex.ac.uk/tex-archive/macros/latex/contrib/geometry/geometry.pdf
% and http://en.wikibooks.org/wiki/LaTeX/Document_Structure
\usepackage{geometry}
% For header and footer
% See: ftp://ctan.tug.org/tex-archive/macros/latex/contrib/fancyhdr/fancyhdr.pdf
% and http://en.wikibooks.org/wiki/LaTeX/Document_Structure
\usepackage{fancyhdr}
% For prettier tables
% See: http://ctan.mackichan.com/macros/latex/contrib/booktabs/booktabs.pdf
% and  http://en.wikibooks.org/wiki/LaTeX/Tables
\usepackage{booktabs}


%%%%%%%%%%%%%%%%%%%%%%%%%%% SETUP %%%%%%%%%%%%%%%%%%%%%%%%%%%

% Set the margins of the paper. By default LaTeX uses huge margins
\geometry{includeheadfoot, margin=2.5cm}
% you can also use
% \geometry{a4paper}
% End of margins setup

% Setup header and footer
% Headers
\pagestyle{fancy} % To get the header and footer
\lhead{\small \textsc{Lab Exercises}}
\rhead{\small \textsc{Introduction to Electronic Engineering}}
% Footers
%\lfoot{Left footer text}
%\cfoot{\thepage} % This is the default behaviour
%\rfoot{Right footer text}

% If you don't want a line below the header or above the footer, 
% change the appropriate header/footerrulewidth to 0pt
\setlength{\headheight}{15.2pt} % This is set to avoid a warning
\renewcommand{\headrulewidth}{0.4pt}
\renewcommand{\footrulewidth}{0.4pt}
% End of header and footer setup

% Setup Problem/Solution environments
\theoremstyle{plain}
\newtheorem{problem}{Problem}[section]

\theoremstyle{remark}
\newtheorem*{solution}{Solution}
% End of Problem/Solution environments setup

%%%%%%%%%%%%%%%%%%%%%%%% END OF SETUP %%%%%%%%%%%%%%%%%%%%%%%%

\title{{\large \bf \MakeUppercase{Introduction to Electronic Engineering}} \\
		\large \textsc{Lab exercises}}

\author{\textsc{\small Antal János Monori \& Emil Már Einarsson }}

\begin{document}
	\maketitle
	\large \bf{\textsc{\section{Lab 1}}
	\begin{problem}
		Problem 1 here.
	\end{problem}

	\begin{solution}
		Solution goes here.
	\end{solution}
	
	\begin{problem}
		Problem 1 here.
	\end{problem}
	
	\begin{solution}
		Solution goes here.
	\end{solution}
	
	\pagebreak

	\large \bf{\textsc{\section{Lab 2}}
	\begin{problem}
		Problem 1 here.
	\end{problem}
	
	\begin{solution}
		Solution goes here.
	\end{solution}
	
	\begin{problem}
		Problem 1 here.
	\end{problem}
	
	\begin{solution}
		Solution goes here.
	\end{solution}
	
	\pagebreak
	
	\large \bf{\textsc{\section{Lab 3}}
	\begin{problem}
		Problem 1 here.
	\end{problem}
	
	\begin{solution}
		Solution goes here.
	\end{solution}
	
	\begin{problem}
		Problem 1 here.
	\end{problem}
	
	\begin{solution}
		Solution goes here.
	\end{solution}
	
	\pagebreak
	
	\large \bf{\textsc{\section{Lab 4}}
	\begin{problem}
		Problem 1 here.
	\end{problem}
	
	\begin{solution}
		Solution goes here.
	\end{solution}
	
	\begin{problem}
		Problem 1 here.
	\end{problem}
	
	\begin{solution}
		Solution goes here.
	\end{solution}
\end{document}

