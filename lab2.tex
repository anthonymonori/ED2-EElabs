	\large \bf{\textsc{\section{Lab 2}}
	\begin{problem}
		In this exercise, we would like to build a thermometer using an NTC sensor. The NTC sensor is a resistance that decreases when the temperature increases. Choose an NTC sensor with nominal value of 10K\(\Omega\) (usually given at 25\(\deg\)).
		\newline
		1. Measure the resistance of the NTC using a multimeter at the room temperature (don't touch the sensor when measuring). Let's assume that the room temperature is about 22 \(\deg\).
		\newline
		2. Measure the resistance of the NTC using a multimeter while you keep the sensor in your hands for about 30 seconds. Let's assume that your hands has the temperature of about 35 \(\deg\).
		\newline
		3. Build the following circuit with: \\
		R\(_{1}\): 10K\(\Omega\)(NTC sensor) \\
		R\(_{2}\): 10K\(\Omega\) \\
		R\(_{3}\): 4.7K\(\Omega\) \\
		R\(_{4}\): 1.8K\(\Omega\) \\
		V: 1.45V
		\newline
		4. Calculate the voltage of V\(_{OC}\) for two cases (room temperature and the hand temperature).
		\newline
		5. Measure the voltage of V\(_{OC}\) for two cases and compare the results with your calculations.
		\begin{figure}[h!]
			\centering
			\includegraphics[width=0.5\textwidth]{images/circuit7.png}
		\end{figure}
	\end{problem}
	
	\begin{solution}
		1.
		\newline
		2.
		\newline
		3.
		\newline
		4.
		\newline
		5.
	\end{solution}
	\clearpage
	\begin{problem}
		In the above circuit, remove the sensor and use R\(_{1}\)=1K\(\Omega\), R\(_{2}\)=100K\(\Omega\) and V=20V. Calculate their equivalent resistance.
		\newline
		1. Obtain the Thevenin equivalent of the circuit by experiment. Hint: open circuit voltage and zero input resistance.
		\newline
		2. Calculate the Thevenin equivalent of the circuit and compare the results.
		\newline
		3. Obtain the Nortin equivalent of the circuit by experiment. Hint: short circuit current and zero input resistance.
		\newline
		4. Calculate the Nortin equivalent of the circuit and compare the results.
	\end{problem}
	
	\begin{solution}
		1.
		\newline
		2.
		\newline
		3.
		\newline
		4.
	\end{solution}
	\clearpage
	\begin{problem}
		Now we would like to connect a resistor to the output port of the previous circuit.
		\newline
		1. Using the Thevenin equivalent, find the smallest possible resistor for the output port such that the current in the resistor remains less than 1mA.
		\newline
		2. Find and connect the resistance that maximizes the power consumption in the output port.
		\newline
		3. Try several resistors that are higher or lower than your findings, then fill the following table and plot the power curve for a variation of resistors. Compare the results with your calculations.
	\end{problem}
	
	\begin{solution}
		1.
		\newline
		2.
		\newline
		3.
		\begin{table}[h]
			\begin{tabular}{| l | l | l | l | l | l | l | l | l | l | l | l |}
				\hline
				\textbf{Try} & 1 & 2 & 3 & 4 & 5 & 6 & 7 & 8 & 9 & 10 & 11 \\ \hline
				Resistance (Ohms) & & & & & & R\(_{max pow}\) & & & & & \\ \hline
				Voltage (V) & & & & & & & & & & & \\ \hline
				Power (mW) & & & & & & & & & & & \\ \hline
			\end{tabular}
		\end{table}
	\end{solution}